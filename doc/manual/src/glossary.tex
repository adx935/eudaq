% !TEX root = EUDAQUserManual.tex
\renewcommand\newacronym[3]{
  \newglossaryentry{#1}{
    type=\acronymtype,
    name={#1},
    description={#2#3},
    text={#1},
    first={#2 (#1)},
    plural={#1\glspluralsuffix},
    firstplural={#2\glspluralsuffix{} (#1\glspluralsuffix)},
  }
}
\newacronym{EUDRB}{teawt}{}
\newacronym{FSM}{finite-state machine}{}
\newacronym{BORE}{beginning-of-run-event}{, basically a run header}
\newacronym{EORE}{end-of-run-event}{, basically a run trailer}
\newacronym{TLU}{the Trigger Logic Unit}{}
\newacronym{NI}{the National Instrument system}{, for reading out the Mimosa 26 sensors}
\newacronym{LCIO}{Linear Collider I/O}{, the file format used by the analysis software}
\newacronym{DUT}{device under test}{}
\newacronym{CDS}{correlated double sampling}{, when two frames are acquired, one before and one after the trigger, and then subtracted to get the actual signal}
